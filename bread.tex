\chapter{Breads}
\label{breads}
\epigraph{``How can a nation be called great if its bread tastes like kleenex?"}{--- \textup{Julia Child}}

The industrial era marked a transformative period in the history of bread production, ushering in technological advancements, changing consumption patterns, and the globalization of this essential food source. The following text delves into the concise evolution of breadmaking, highlighting key developments and global implications.

\textbf{Technological Advancements:} The industrial revolution, beginning in the 18th century, revolutionized bread production. Mechanization replaced labor-intensive methods, introducing steam-powered mills for finely ground flour. This led to the mass production of lighter, more consistent bread. Mechanized dough mixers, fermentation chambers, and ovens streamlined the entire process.

\textbf{Commercial Bread Production:} The emergence of large-scale commercial bakeries was a direct outcome of industrialization. Urban centers saw the rise of bakeries, providing more accessible bread. Automation improved packaging and distribution, extending shelf life and expanding distribution networks. This shift from home baking to commercial production reshaped the global bread industry.

\textbf{Bread Diversity:} Globalization facilitated the diversification of bread varieties, each influenced by regional ingredients and traditions. France popularized the baguette, while Italy introduced ciabatta. India showcased naan and chapati. This diversification exemplified bread's adaptability to local cuisines.

\textbf{Nutritional Considerations:} As refined white bread gained prominence due to longer shelf life and a smoother texture, concerns emerged regarding its nutritional value. This led to the development of whole-grain and multigrain bread, preserving grain nutrients and dietary fiber to address health concerns.

\textbf{Bread and Globalization:} Bread transcended cultural boundaries, becoming a global commodity. Fusion bread varieties, such as the Mexican bolillo, emerged as the result of cultural exchange. Sandwiches, utilizing bread as a versatile carrier for fillings, became a global phenomenon, solidifying bread's role in contemporary diets.

\textbf{Contemporary Trends:} Modern bread production has witnessed a resurgence of interest in artisanal and craft baking. Consumers seek locally sourced, handcrafted bread with distinct flavors and textures, reflecting a desire for authenticity and traditional baking techniques. The future of bread production is influenced by sustainability and technological innovations. Researchers explore alternative ingredients, like insect-based flours, to address environmental and nutritional challenges. Automation and digital technology advancements are poised to further streamline bread production processes.

The evolution of bread production since the industrial era is marked by technological progress, changing consumer preferences, and globalization. From mechanization to the rich diversity of bread varieties worldwide, bread remains a foundational component of global cuisine. Its future promises innovation and sustainability, ensuring its enduring role in modern diets.

\clearpage
\section{Rustic Bread}
\label{rusticbread}
Experience the magic of our Homemade Wood-Fired Rustic Bread. Its golden-brown, smoky crust and airy, herb-infused interior create a culinary masterpiece. This versatile bread elevates any meal, from appetizers with olive oil and balsamic vinegar to hearty soups and stews. Become a master baker at home and savor tradition's warmth, one slice at a time.

This recipe makes \emph{4 servings}. It takes about \emph{4 hours} and has an \emph{intermediate} level of difficulty. 

\paragraph{Items}
\begin{itemize}
	\item For the bread:
	\begin{itemize}[noitemsep]
	    \item[\ding{182}] 4 cups all-purpose flour.
	    \item[\ding{183}] 1 $\frac{1}{2}$ teaspoons salt.
	    \item[\ding{184}] 1 $\frac{1}{2}$ teaspoons active dry yeast
	    \item[\ding{185}] 2 cups lukewarm water (around 43°C).
	\end{itemize}
	\item For the seasoning:
	\begin{itemize}[noitemsep]
	    \item[\ding{182}] 2 tablespoons olive oil.
	    \item[\ding{183}] 1 tablespoon dried rosemary.
	    \item[\ding{184}] 1 tablespoon dried thyme.
	    \item[\ding{185}] 1 teaspoon garlic powder.
	    \item[\ding{186}] $\frac{1}{2}$ teaspoon smoked paprika.
	    \item[\ding{187}] $\frac{1}{2}$ teaspoon black pepper.
	    \item[\ding{188}] $\frac{1}{2}$ teaspoon sea salt flakes.
	\end{itemize}
\end{itemize}

\paragraph{Instruction}
\begin{enumerate}[noitemsep]
	\item \textbf{Activate the yeast:} In a small bowl, combine the lukewarm water and yeast. Let it sit for about 5-10 minutes until it becomes frothy. This indicates that the yeast is active and ready to use.
	\item \textbf{Mix the dry ingredients:} In a large mixing bowl, combine the all-purpose flour and salt. Create a well in the center of the flour mixture.
	\item \textbf{Combine the wet and dry ingredients:} Pour the activated yeast mixture into the well you created in the flour. Stir until a shaggy dough forms.
	\item \textbf{Knead the dough:} Turn the dough out onto a floured surface and knead it for about 10 minutes until it becomes smooth and elastic. You can add a little more flour if the dough is too sticky.
	\item \textbf{First rise:} Place the dough in a lightly oiled bowl, cover it with a clean kitchen towel or plastic wrap, and let it rise in a warm, draft-free place for about 1-2 hours, or until it has doubled in size.
	\item \textbf{Preheat the oven:} If you have a pizza stone or baking stone, place it in your oven and preheat it to 230°C. If you don't have a stone, you can use a cast-iron skillet or a heavy baking sheet.
	\item \textbf{Shape the dough:} Punch down the risen dough and shape it into a round or oval loaf. Place it on a piece of parchment paper.
	\item \textbf{Season the bread:} In a small bowl, mix together the olive oil, dried rosemary, dried thyme, garlic powder, smoked paprika, and black pepper. Brush this mixture generously over the surface of the bread. Sprinkle sea salt flakes on top.
	\item \textbf{Second rise:} Cover the bread with a kitchen towel and let it rise for an additional 30 minutes while your oven continues to preheat.
	\item \textbf{Bake:} Carefully transfer the bread (along with the parchment paper) onto the preheated stone or skillet in the oven. Bake for 30-35 minutes or until the bread is golden brown and sounds hollow when tapped on the bottom.
	\item \textbf{Cool:} Allow the bread to cool on a wire rack for at least 30 minutes before slicing. This will help the interior set properly.
\end{enumerate}
Enjoy your homemade rustic bread with the flavors of a wood-fired oven! It's perfect for serving with olive oil and balsamic vinegar or as a side with soups and stews.
\clearpage
