\chapter{Breads}
\label{breads}
\epigraph{``How can a nation be called great if its bread tastes like kleenex?"}{--- \textup{Julia Child}}

The industrial era, spanning the late 18th and 19th centuries, catalyzed a pivotal shift in bread production. It brought forth technological innovations, altered consumption habits, and facilitated the worldwide diffusion of this fundamental foodstuff. This succinct overview explores the evolution of breadmaking during this era, focusing on pivotal milestones and their global ramifications.

\textbf{Technological Advancements:} The Industrial Revolution, commencing in the 18th century, heralded a groundbreaking transformation in bread production. It supplanted labor-intensive techniques with steam-powered mills, particularly the influential roller milling system developed by John Stevens in the early 19th century, enabling the production of finer, more refined flour. This pivotal advancement facilitated the mass production of bread characterized by greater consistency and lighter texture.

Moreover, mechanization extended beyond milling. It encompassed the entire breadmaking process, with mechanized dough mixers, fermentation chambers, and ovens. These innovations streamlined and standardized bread production, further enhancing efficiency and product quality.

\textbf{Commercial Bread Production:} The proliferation of large-scale commercial bakeries directly resulted from the industrialization wave. In urban hubs, the ascendance of bakeries made bread more accessible to a burgeoning population. Automation not only elevated efficiency in production but also enhanced packaging and distribution capabilities, significantly extending the shelf life of bread and broadening distribution networks. This transition from home-based baking to industrial-scale production wrought a profound transformation in the global bread industry.

\textbf{Bread Diversity:} Globalization played a pivotal role in expanding the array of bread varieties, with each type reflecting regional ingredients and culinary traditions. For instance, France became renowned for the baguette, while Italy introduced the ciabatta. In India, naan and chapati took center stage. This diversification exemplified bread's remarkable adaptability to suit diverse local cuisines, underscoring its role as a versatile and culturally resonant staple.

\textbf{Nutritional Considerations:} As refined white bread gained prominence due to longer shelf life and a smoother texture, concerns emerged regarding its nutritional value. This led to the development of whole-grain and multigrain bread, preserving grain nutrients and dietary fiber to address health concerns.

\textbf{Bread and Globalization:} Bread's universal appeal transcended cultural borders, establishing itself as a global staple. It fostered fusion bread types like the Mexican bolillo, borne from cultural interchange. Additionally, sandwiches, leveraging bread's versatility as a canvas for a diverse range of fillings, attained worldwide prominence, cementing bread's integral position in modern diets across the globe.

\textbf{Contemporary Trends:} In contemporary bread production, there's a notable revival of interest in artisanal and craft baking. Consumers increasingly crave locally sourced, handcrafted bread, known for its unique flavors and textures. This preference underscores a broader trend towards authenticity and traditional baking methods.

The trajectory of bread production in the future is significantly influenced by sustainability and technological advancements. Researchers are actively exploring alternative ingredients, including insect-based flours, to tackle environmental and nutritional challenges. Automation and digital technology are poised to further streamline bread production processes.

This evolution of bread production, tracing its roots from the industrial era to the present, showcases a journey of technological advancement, shifting consumer tastes, and globalization. From the early mechanization of bread production to the rich diversity of bread varieties worldwide, bread has maintained its integral status in global cuisine. Its future promises continued innovation and sustainability, ensuring its enduring role in modern diets.

\section{Rustic Bread}
\label{rusticbread}
Indulge in the enchanting experience of our Homemade Rustic Bread, celebrated for its golden-brown, smoky crust and airy, herb-infused interior. This versatile bread enhances a variety of dishes, from simple appetizers with olive oil and balsamic vinegar to hearty soups and stews. It's a slice of tradition's warmth, allowing you to embrace the art of master baking in the comfort of your own home.

This recipe makes \emph{4 servings}. It takes about \emph{4 hours} and has an \emph{intermediate} level of difficulty. 

\paragraph{Items}
\begin{itemize}
	\item For the bread:
	\begin{itemize}[noitemsep]
	    \item[\ding{182}] 4 cups all-purpose flour.
	    \item[\ding{183}] 1 $\frac{1}{2}$ teaspoons salt.
	    \item[\ding{184}] 1 $\frac{1}{2}$ teaspoons active dry yeast
	    \item[\ding{185}] 2 cups lukewarm water (around 43°C).
	\end{itemize}
	\item For the seasoning:
	\begin{itemize}[noitemsep]
	    \item[\ding{182}] 2 tablespoons olive oil.
	    \item[\ding{183}] 1 tablespoon dried rosemary.
	    \item[\ding{184}] 1 tablespoon dried thyme.
	    \item[\ding{185}] 1 teaspoon garlic powder.
	    \item[\ding{186}] $\frac{1}{2}$ teaspoon smoked paprika.
	    \item[\ding{187}] $\frac{1}{2}$ teaspoon black pepper.
	    \item[\ding{188}] $\frac{1}{2}$ teaspoon sea salt flakes.
	\end{itemize}
\end{itemize}

\paragraph{Instruction}
\begin{enumerate}[noitemsep]
	\item \textbf{Activate the yeast:} In a small bowl, combine the lukewarm water and yeast. Let it sit for about 5-10 minutes until it becomes frothy. This indicates that the yeast is active and ready to use.
	\item \textbf{Mix the dry ingredients:} In a large mixing bowl, combine the all-purpose flour and salt. Create a well in the center of the flour mixture.
	\item \textbf{Combine the wet and dry ingredients:} Pour the activated yeast mixture into the well you created in the flour. Stir until a shaggy dough forms.
	\item \textbf{Knead the dough:} Turn the dough out onto a floured surface and knead it for about 10 minutes until it becomes smooth and elastic. You can add a little more flour if the dough is too sticky.
	\item \textbf{First rise:} Place the dough in a lightly oiled bowl, cover it with a clean kitchen towel or plastic wrap, and let it rise in a warm, draft-free place for about 1-2 hours, or until it has doubled in size.
	\item \textbf{Preheat the oven:} If you have a pizza stone or baking stone, place it in your oven and preheat it to 230°C. If you don't have a stone, you can use a cast-iron skillet or a heavy baking sheet.
	\item \textbf{Shape the dough:} Punch down the risen dough and shape it into a round or oval loaf. Place it on a piece of parchment paper.
	\item \textbf{Season the bread:} In a small bowl, mix together the olive oil, dried rosemary, dried thyme, garlic powder, smoked paprika, and black pepper. Brush this mixture generously over the surface of the bread. Sprinkle sea salt flakes on top.
	\item \textbf{Second rise:} Cover the bread with a kitchen towel and let it rise for an additional 30 minutes while your oven continues to preheat.
	\item \textbf{Bake:} Carefully transfer the bread (along with the parchment paper) onto the preheated stone or skillet in the oven. Bake for 30-35 minutes or until the bread is golden brown and sounds hollow when tapped on the bottom.
	\item \textbf{Cool:} Allow the bread to cool on a wire rack for at least 30 minutes before slicing. This will help the interior set properly.
\end{enumerate}
Enjoy your homemade rustic bread with the flavors similar to a wood-fired oven! It's perfect for serving with olive oil and balsamic vinegar or as a side with soups and stews.
\clearpage
