\chapter{Desserts}
\label{desserts}
\epigraph{``Desserts are the fairy tales of the kitchen -- a happily-ever-after to supper."}{--- \textup{Terri Guillemets}}

Desserts, those delectable sweet treats enjoyed at the conclusion of a meal or as an indulgent stand-alone pleasure, have a rich and complex history that has unfolded over centuries, with their development from the Renaissance period onwards being particularly intriguing. This essay explores the multifaceted journey of desserts, encompassing cultural, technological, and economic factors, from the Renaissance to the present day.

\paragraph{The Renaissance and the Sweetening of Europe:} The Renaissance, spanning from the 14th to the 17th century, was marked by profound transformations in European culture, and the world of desserts was no exception. Sugar, a prized and costly commodity in the Middle Ages, became more accessible and affordable during this period. The Age of Exploration brought sugar from the Middle East and the New World to Europe, resulting in a shift in culinary practices. Italian and French pastry chefs took the lead in crafting intricate sugar sculptures, marzipan creations, and sweet delicacies, laying the foundation for the artistry that would define European desserts in the centuries to come.

\paragraph{Desserts as Royal Indulgences:} Desserts quickly evolved into symbols of luxury and prestige in the courts of Europe. Monarchs and nobility maintained their own pastry chefs, who excelled in the creation of elaborate dessert presentations. Notably, the French court set the standard for opulent dessert courses known as "entremets", consisting of a dazzling array of sweet confections.

\paragraph{Chocolate's European Arrival:} One of the most influential developments of the Renaissance era was the introduction of chocolate to Europe. Spanish explorers returned from the Americas with cocoa beans, which were initially consumed as a beverage, and sweetened with sugar to make it more palatable. Eventually, cocoa-based solid confections were developed, setting the stage for the emergence of the chocolate industry.

\paragraph{The Birth of Ice Cream:} During the late 16th century, Italian chefs created the early versions of ice cream, primarily reserved for royal feasts. The fundamental principles of ice cream, including a mixture of milk, sugar, and flavorings, were established during this period, forming the foundation for the frozen dessert that is beloved worldwide today.

\paragraph{The Global Exchange of Dessert Ingredients:} The age of European exploration and colonial expansion brought forth an exchange of ingredients crucial to dessert making. This global exchange of sugar, spices, and other culinary components facilitated the fusion of culinary traditions and the development of new desserts worldwide.

\paragraph{The Industrial Revolution and Mass Production:} The 18th and 19th centuries ushered in significant technological advancements, including the invention of ice cream machines and the mass production of sweets. This accessibility made desserts more affordable and widespread, leading to the establishment of dessert-focused businesses and a more democratic enjoyment of these sweet pleasures.

\paragraph{Diverse Cultural Contributions:} Around the world, different cultures have made significant contributions to the realm of desserts. For example, in India, the Mughal Empire introduced sweets such as gulab jamun and jalebi, which have transcended their regional origins to become internationally beloved delicacies. Similarly, the Middle East's baklava and Turkish delight have achieved global recognition.

\paragraph{The American Dessert Revolution:} In the 20th century, the United States played a pivotal role in the transformation of the dessert landscape. Innovations such as the chocolate chip cookie, the ice cream sundae, and the establishment of iconic brands like Hershey's and Nestlé reshaped the way people enjoyed sweets.

\paragraph{The Health-Conscious Movement:} In recent decades, growing concerns about the health implications of excessive sugar and fat consumption have led to the development of healthier dessert alternatives. Sugar-free and vegan dessert options have emerged as a response to these health-conscious trends.

\paragraph{The Digital Age and Dessert Trends:} In the 21st century, the influence of social media and food blogs on dessert culture is undeniable. Desserts have become a form of artistic expression, with a strong emphasis on visually appealing creations that are shared on platforms like Instagram. Innovative desserts such as cronuts (croissant-doughnut hybrids) and macarons have gained immense popularity due to their photogenic aesthetics.

\paragraph{Global Dessert Fusion:} Today, the culinary world reflects globalization, resulting in a captivating fusion of dessert traditions. Elements of Asian, European, and American dessert styles can be found in sweet creations worldwide, exemplified by the popularity of matcha-flavored pastries, bubble tea, and churro ice cream sandwiches.

In conclusion, the development of desserts from the Renaissance to the present day is a testament to the enduring allure and dynamic nature of culinary arts and human culture. What began as simple, sweet dishes in the Renaissance has evolved into a complex and diverse array of confections, serving as a form of cultural expression, art, and celebration. The history of desserts is a reflection of the ever-evolving global culinary landscape and the rich tapestry of human creativity in satisfying our collective sweet tooth.

\section{Grandma's Polish Doughnuts}
\label{grandmaspolishdoughnuts}
Delve into the enchanting world of Grandma's Polish doughnuts, a treasured recipe passed down through generations, infusing joy and sweetness into countless family gatherings. These delectable treats, known as "pączki" in Polish, offer a multisensory delight. Imagine biting into a soft, pillowy doughnut with a golden exterior yielding to a melt-in-your-mouth interior. The irresistible aroma of freshly fried dough tantalizes the senses. Each bite is sheer bliss, generously filled with homemade fruit preserves or velvety custard.

Grandma's secret recipe harmoniously balances sweetness, complemented by a delicate sugar dusting that imparts a satisfying crunch to the tender dough. Be it classic raspberry, tangy apricot, or rich vanilla custard, the flavors are divine. Crafting these doughnuts entails a labor of love, with meticulous ingredient mixing to ensure the dough's signature lightness. Shaping each doughnut into plump, round morsels is an anticipation-filled process.

With a gentle sizzle, they achieve golden perfection through frying, creating a crispy shell that encases the heavenly filling. The first bite transcends you to a world of sweet nostalgia, evoking cherished moments spent with loved ones at family gatherings. These Polish doughnuts represent more than a mere treat; they embody love, tradition, and the joy of sharing something profoundly special. Gather your loved ones, don your apron, and embark on a culinary adventure with Grandma's Polish doughnuts.

This recipe makes \emph{15 servings}. It takes about \emph{2 $\frac{1}{2}$ hours} and has an \emph{intermediate} level of difficulty. 

\paragraph{Items}
\begin{itemize}[noitemsep]
    \item[\ding{182}] 8 egg yolks.
    \item[\ding{183}] 1kg of flower.
    \item[\ding{184}] 300g sugar.
    \item[\ding{185}] 300g butter.
    \item[\ding{186}] 400ml milk (3,5\% fat).
    \item[\ding{187}] 1 shot spiritus.
    \item[\ding{188}] 100g yeast.
    \item[\ding{189}] Lemon and vanilla oil.
    \item[\ding{190}] Salt and plum jam.
\end{itemize}

\paragraph{Instruction}
To craft these delectable Polish doughnuts, start by heating a small pot over medium heat and pouring in the milk. Sweeten it with a tablespoon of sugar, and add two tablespoons of flour to create a yeast-friendly base. Ensure the milk is lukewarm, not boiling. Dissolve the yeast into the warm milk, traditionally done by hand, and let it sit at room temperature for 30 minutes to allow yeast growth.

In a large bowl, combine the remaining flour, egg yolk, sugar, a tablespoon of citrus oil, and vanilla oil. Add the yeast-infused milk mixture and knead into a dough. Gradually integrate a tablespoon of butter while kneading. Finish by adding spirit and kneading for a minimum of five minutes until the dough reaches a core temperature of around 30°C. Let the dough rise for at least two hours.

After the rising period, prepare a floured work surface and shape the dough, which you've allowed to rise under a damp cloth, into doughnuts about the size of tennis balls. Heat a pot with neutral-tasting oil (canola or sunflower oil work well) until it bubbles upon adding dough. Press the dough down with your thumb, fill each with about a heaping teaspoon of plum jam (avoid overfilling), and seal them back into a ball shape. Fry in the hot oil until golden brown.

For the finishing touch, roll the doughnuts in regular sugar (not powdered) before draining. After about a minute, they're ready to be relished. Traditional fillings include blackberry or plum jam, while cherries, strawberries, or rosehip can also be used seasonally or depending on regional preferences. Enjoy!