\chapter{Desserts}
\label{desserts}
\epigraph{``Desserts are the fairy tales of the kitchen -- a happily-ever-after to supper."}{--- \textup{Terri Guillemets}}

\clearpage
\section{Grandma's Polish Doughnuts}
\label{grandmaspolishdoughnuts}
Experience the enchanting realm of Grandma's Polish doughnuts, a cherished recipe passed down through generations, bringing joy and sweetness to countless family gatherings. Known as "pączki" in Polish, these delectable treats are a sensory delight. Picture sinking your teeth into a soft, pillowy doughnut, its golden exterior yielding to a melt-in-your-mouth interior. The irresistible aroma of freshly fried dough fills the air, tempting your senses. Each bite is pure bliss, generously filled with homemade fruit preserves or velvety custard. Grandma's secret recipe ensures a perfect balance of sweetness, with a delicate sugar dusting that adds a satisfying crunch to the tender dough. Whether you choose classic raspberry, tangy apricot, or rich vanilla custard, the flavors are simply divine. Preparing these doughnuts is a labor of love, carefully mixing ingredients to allow the dough to rise and achieve its signature lightness. Shaping each doughnut into plump, round morsels is a delightful anticipation. With a gentle sizzle, they are fried to golden perfection, creating a crispy shell that encases the heavenly filling. The first bite transports you to a world of sweet nostalgia, evoking cherished moments spent with loved ones at family gatherings. These Polish doughnuts are more than a treat; they symbolize love, tradition, and the joy of sharing something truly special. So gather your loved ones, don your apron, and embark on a culinary adventure with Grandma's Polish doughnuts.

This recipe makes \emph{15 servings}. It takes about \emph{2 $\frac{1}{2}$ hours} and has an \emph{intermediate} level of difficulty. 

\paragraph{Items}
\begin{itemize}[noitemsep]
    \item[\ding{182}] 8 egg yolks.
    \item[\ding{183}] 1kg of flower.
    \item[\ding{184}] 300g sugar.
    \item[\ding{185}] 300g butter.
    \item[\ding{186}] 400ml milk (3,5\% fat).
    \item[\ding{187}] 1 shot spiritus.
    \item[\ding{188}] 100g yeast.
    \item[\ding{189}] Lemon and vanilla oil.
    \item[\ding{190}] Salt and plum jam.
\end{itemize}

\paragraph{Instruction}
First, we heat a small pot over medium heat and add the entire milk. We take a tablespoon of sugar and sweeten the warm milk with it. Additionally, we add two tablespoons of flour to the milk to form the nutritional base for the yeast. Now, we turn off the heat. The milk should be lukewarm and by no means boiling. In this lukewarm milk, we dissolve the entire yeast while stirring briefly, preferably with our hands, as is traditionally customary. We let the milk sit for 30 minutes at room temperature to allow the yeast to grow.

Next, we need a large bowl into which we add the remaining flour, the egg yolk, the sugar, and a tablespoon of citrus oil as well as vanilla oil. Finally, we add the milk mixture with the yeast and carefully knead everything into a dough. The resulting dough is then enriched with fat. We continue kneading and after 2-3 minutes of kneading, we add a tablespoon of butter until the butter is fully incorporated. Lastly, we add the spirit and knead the dough carefully again, for at least five minutes, until it reaches a core temperature of about 30°C. We let this dough rise for at least 2 hours. After the time has elapsed, we can prepare a work surface.

We dust the work surface with flour and shape the matured and risen dough, which we had covered with a damp cloth to allow it to rise (this is just an option, but I like to do it this way, and so does Grandma), into our doughnuts. Each doughnut should be about the size of a tennis ball. At the same time, we can heat a large pot with hot oil for frying, achieving the tastiest results with neutral-tasting oil such as canola or sunflower oil. Once the oil is hot enough to immediately start bubbling when dough is added, we are ready to fry the doughnuts. Using a thumb, we press them down and then fill them with plum jam, about a heaping teaspoon, but be careful not to overdo it, or the filling will spill out! Traditional fillings for Polish doughnuts are blackberry or plum jam, but depending on the season, cherries or strawberries can also be used. In Bavaria, they would use rosehip! The doughnuts need to be sealed back into a ball shape and then fried in the oil until golden brown. For decoration, regular sugar, not powdered sugar, is used. Simply roll the finished doughnut in a plate full of sugar before draining. Place them in a draining dish, and after about a minute, the doughnuts are ready to be enjoyed.