\chapter{Main Courses}
\label{mains}
\epigraph{``Cooking requires confident guesswork and improvisation -- experimentation and substitution, dealing with failure and uncertainty in a creative way."}{--- \textup{Paul Theroux}}

\section{Grandma's Lentil Stew}
\label{grandmaslentilsstew}

This recipe is a simple stew that is particularly affordable to prepare. The stew can be enjoyed as a hot or cold meal, especially during the cold season. Optionally, you can add a knob of ginger, cooked along with the stew, a few cloves of garlic, according to taste, or one to two chili peppers to give it a different flavor profile. However, let's start with the original recipe. In Germany, this is a traditional recipe from the rural bourgeois cuisine and is therefore called Grandma's Lentil Stew.

This recipe makes about \emph{6-8 servings}. It takes about \emph{45 minutes} and has an \emph{easy} level of difficulty. 

\paragraph{Items}
\begin{itemize}[noitemsep]
	\item[\ding{182}] $\frac{1}{4}$ celery.
	\item[\ding{183}] 1 leek.
	\item[\ding{184}] 1 handful of parsley.
	\item[\ding{185}] 2 carrots.
	\item[\ding{186}] 2 slices of bacon.
	\item[\ding{187}] 4 onions.
	\item[\ding{188}] 500g lentils.
	\item[\ding{189}] Salt and black pepper.
\end{itemize}

\paragraph{Instruction} 
We start by cutting the bacon into large, rough cubes. The two slices should weigh no less than 200g. We then add these cubes to a large pot and place it on the stove over high heat. It is important to let the bacon roast properly until it becomes crispy. The fat should render, which is responsible for binding the stew and adds flavor. If desired, you can also add some back fat. Additionally, you can optionally add 50g of butter, which gives the stew a nutty flavor. However, make sure to add the butter only when the bacon is already close to being crispy.

While the bacon is frying in the pot, we cut the vegetables -- celery, carrots, onions, and parsley. When cutting, everything should be diced as small as possible, with a side length of slightly less than half a centimeter. This ensures that the flavors of the different ingredients are well distributed in the stew and mixed with each bite, without any one flavor dominating.

First, we add the onions to the bacon, once a small amount of fat has already rendered. We let them become golden brown. Once the bacon is crispy and the onions are golden brown, we add the lentils. These are also roasted for about five minutes. It is perfectly fine if some sediment forms at the bottom of the pot. This adds flavor through roasting aromas.

After the five minutes, we add one liter of water for 500g of lentils. Optionally, you can add two tablespoons of vegetable broth to the water. The stew is already taking shape and is now cooked over high heat for 15 minutes. After the 15 minutes, we add the carrots, celery, and half of the parsley, and cook everything for another 10 minutes. Then we add the leek and the second half of the parsley, and cook everything for another 10 minutes. It is advisable to cover the stew with a lid, as it does not need to reduce. The lentils will absorb almost all of the water. If needed, you can add more water or water mixed with broth. Finally, generously season with black pepper and salt to taste.

As a serving suggestion, a hot Bockwurst and some bread garnished with additional parsley go well with the stew. However, it can also be enjoyed without any additional ingredients. Enjoy your meal!

\section{Cheese Soufflé}
\label{cheesesouffle}
Our Cheese Soufflé is a classic French dish that offers a delightful experience for your taste buds. It has a golden, puffed exterior with a delicate crispness and a creamy interior. The flavors of Gruyère or Swiss cheese, balanced with Parmesan, create a rich and nutty taste. With the addition of black pepper and nutmeg, each spoonful is a symphony of flavors. The soufflé's lightness comes from perfectly whisked egg whites, resulting in a melt-in-your-mouth texture. Whether enjoyed as an appetizer or for brunch, our Cheese Soufflé is a culinary masterpiece that will leave you wanting more. Get ready to be enchanted by the magic of this timeless dish as it reaches new levels of deliciousness.

This recipe makes about \emph{2 servings}. It takes about \emph{50 minutes} and has an \emph{intermediate} level of difficulty. 

\paragraph{Items}
\begin{itemize}[noitemsep]
	\item[\ding{182}] 4 tablespoons unsalted butter, \\ plus extra for greasing the dish.
	\item[\ding{183}] $\frac{1}{4}$ cup grated Parmesan cheese.
	\item[\ding{184}] 1 cup milk.
	\item[\ding{185}] 4 tablespoons all-purpose flour.
	\item[\ding{186}] $\frac{1}{2}$ teaspoon salt.
	\item[\ding{187}] $\frac{1}{4}$ teaspoon black and cayenne pepper.
	\item[\ding{188}] $\frac{1}{4}$ teaspoon nutmegpepper.
	\item[\ding{189}] 1 $\frac{1}{2}$ cups grated Gruyère or Swiss cheese.
	\item[\ding{190}] 4 large eggs, separated.
	\item[\ding{191}] Pinch of cream of tartar.	
\end{itemize}

\paragraph{Instruction} Preheat your oven to 190°C. Butter a 2-quart soufflé dish and coat it with grated Parmesan cheese, tapping out any excess.

In a medium saucepan, melt the butter over medium heat. Stir in the flour and cook for about 2 minutes, until the mixture turns slightly golden. Gradually whisk in the milk and continue cooking, whisking constantly, until the mixture thickens and comes to a boil. Remove from heat. Stir in the salt, black pepper, cayenne pepper, nutmeg, and grated Gruyère or Swiss cheese. Mix until the cheese is melted and the mixture is smooth.

In a separate bowl, beat the egg yolks until smooth. Gradually whisk in a small amount of the cheese mixture to temper the yolks, then pour the tempered yolks into the saucepan with the remaining cheese mixture. Stir until well combined.

In a clean mixing bowl, beat the egg whites and cream of tartar with an electric mixer on medium speed until stiff peaks form. Gently fold about one-third of the beaten egg whites into the cheese mixture to lighten it. Then, carefully fold in the remaining egg whites until no streaks remain. Pour the mixture into the prepared soufflé dish, smoothing the top with a spatula. Run your thumb around the inside edge of the dish to create a small groove, which will help the soufflé rise evenly. Place the soufflé dish on a baking sheet and bake in the preheated oven for about 25--30 minutes, or until the soufflé is puffed, golden brown, and set in the center.

Serve immediately, as soufflés tend to deflate quickly. Enjoy the light and fluffy cheese soufflé! You can customize this recipe by adding other ingredients like cooked vegetables, ham, or herbs to the cheese mixture before folding in the egg whites.

\section{Tomato Soup}
\label{tomatosoup}
Indulge in the ultimate comfort food -- a steaming bowl of tomato soup bursting with the rich flavors of ripe tomatoes. This velvety soup is a symphony of tangy sweetness and savory goodness. Imagine the mouthwatering aroma of fresh tomatoes simmering with aromatic herbs and spices. Each spoonful offers a delightful balance of flavors, with the natural sweetness of tomatoes enhanced by garlic and onions. The tanginess is complemented by a touch of acidity, creating a zesty kick. The soup reaches new heights with a swirl of creamy goodness, adding a luxurious velvety texture. The creaminess perfectly balances the acidity, creating a harmonious marriage of flavors. Whether enjoyed as a comforting meal or an elegant starter, tomato soup is a timeless classic that never fails to satisfy. So, grab a spoon and let the flavors transport you to a world of culinary delight.

This recipe makes about \emph{6-8 servings}. It takes about \emph{1 $\frac{1}{2}$ hours} and has an \emph{easy} level of difficulty. 

\paragraph{Items}
\begin{itemize}[noitemsep]
	\item[\ding{182}] 2kg vine tomatoes.
	\item[\ding{183}] 1 can of peeled or chopped tomatoes.
	\item[\ding{184}] 1 can of coconut milk.
	\item[\ding{185}] 4 onions.
	\item[\ding{186}] 50g ginger.
	\item[\ding{187}] 1 garlic head.
	\item[\ding{188}] 6 chili peppers.
	\item[\ding{189}] Olive oil.
	\item[\ding{190}] A bunch of cilantro.
	\item[\ding{191}] Cinnamon, honey, salt and pepper.
\end{itemize}

\paragraph{Instruction} First, we preheat the oven to 200°C. Then, we halve the tomatoes, cut a small piece off the garlic head to expose the cloves, and quarter the onions. Next, we place the tomatoes, onions, garlic, ginger, and chilies on a baking sheet. Drizzle generously with olive oil and sprinkle with a pinch of salt. The vegetables are then roasted in the oven for about 30 minutes. The chilies should become black, developing a sweet aroma, while the garlic and ginger become soft and the tomatoes develop a slightly blackened crust on top, which will later add roasted flavors.

Meanwhile, we can prepare a large soup pot and add the canned tomatoes and coconut milk. We then add a generous amount of pepper, a teaspoon of salt, a teaspoon of cinnamon, and a tablespoon of honey. Additionally, we pour in half a liter of tap water and bring everything to a boil. Cook everything on medium heat, ensuring it comes to a boil before reducing the heat to a minimum. After 30 minutes, remove the vegetables from the oven. Peel the ginger and separate the garlic cloves from the head. Add all the vegetables to the pot and pour in another half liter of tap water. Use an immersion blender to carefully puree everything into a creamy, homogeneous liquid. Finely chop the cilantro and add half to the pot. Bring everything to a boil until the desired consistency is reached, reducing the soup slightly. Season with salt and generously add pepper, as tomatoes can handle it well. I recommend serving with roasted garlic bread made with potato bread, as it pairs perfectly. Garnish each plate with a pinch of fresh cilantro, which can be mixed into the soup when eating.