\chapter{Main Courses}
\label{mains}
\epigraph{``Cooking requires confident guesswork and improvisation -- experimentation and substitution, dealing with failure and uncertainty in a creative way."}{--- \textup{Paul Theroux}}

\section{Grandma's Lentil Stew}
\label{grandmaslentilsstew}

This recipe is a simple stew that is particularly affordable to prepare. The stew can be enjoyed as a hot or cold meal, especially during the cold season. Optionally, you can add a knob of ginger, cooked along with the stew, a few cloves of garlic, according to taste, or one to two chili peppers to give it a different flavor profile. However, let's start with the original recipe. In Germany, this is a traditional recipe from the rural bourgeois cuisine and is therefore called Grandma's Lentil Stew.

This recipe makes about \emph{$6$-$8$ servings}. It takes about \emph{$45$ minutes} and has an \emph{easy} level of difficulty. 

\paragraph{Items}
\begin{itemize}[noitemsep]
	\item[\ding{182}] $\frac{1}{4}$ celery.
	\item[\ding{183}] $1$ leek.
	\item[\ding{184}] $1$ handful of parsley.
	\item[\ding{185}] $2$ carrots.
	\item[\ding{186}] $2$ slices of bacon.
	\item[\ding{187}] $4$ onions.
	\item[\ding{188}] $500$g lentils.
	\item[\ding{189}] Salt and black pepper.
\end{itemize}

\paragraph{Instruction} 
We start by cutting the bacon into large, rough cubes. The two slices should weigh no less than $200$g. We then add these cubes to a large pot and place it on the stove over high heat. It is important to let the bacon roast properly until it becomes crispy. The fat should render, which is responsible for binding the stew and adds flavor. If desired, you can also add some back fat. Additionally, you can optionally add $50$g of butter, which gives the stew a nutty flavor. However, make sure to add the butter only when the bacon is already close to being crispy. 

While the bacon is frying in the pot, we cut the vegetables -- celery, carrots, onions, and parsley. When cutting, everything should be diced as small as possible, with a side length of slightly less than half a centimeter. This ensures that the flavors of the different ingredients are well distributed in the stew and mixed with each bite, without any one flavor dominating.

First, we add the onions to the bacon, once a small amount of fat has already rendered. We let them become golden brown. Once the bacon is crispy and the onions are golden brown, we add the lentils. These are also roasted for about five minutes. It is perfectly fine if some sediment forms at the bottom of the pot. This adds flavor through roasting aromas. After the five minutes, we add one liter of water for $500$g of lentils. Optionally, you can add two tablespoons of vegetable broth to the water. The stew is already taking shape and is now cooked over high heat for $15$ minutes. After the $15$ minutes, we add the carrots, celery, and half of the parsley, and cook everything for another $10$ minutes. Then we add the leek and the second half of the parsley, and cook everything for another $10$ minutes. It is advisable to cover the stew with a lid, as it does not need to reduce. The lentils will absorb almost all of the water. If needed, you can add more water or water mixed with broth.

Finally, generously season with black pepper and salt to taste. As a serving suggestion, a hot Bockwurst and some bread garnished with additional parsley go well with the stew. However, it can also be enjoyed without any additional ingredients. Enjoy your meal!
