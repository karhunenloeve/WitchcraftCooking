\chapter{Main Courses}
\label{mains}
\epigraph{``Cooking requires confident guesswork and improvisation -- experimentation and substitution, dealing with failure and uncertainty in a creative way."}{--- \textup{Paul Theroux}}

\section{Grandma's Lentil Stew}
\label{grandmaslentilsstew}

This recipe is a simple stew that is particularly affordable to prepare. The stew can be enjoyed as a hot or cold meal, especially during the cold season. Optionally, you can add a knob of ginger, cooked along with the stew, a few cloves of garlic, according to taste, or one to two chili peppers to give it a different flavor profile. However, let's start with the original recipe. In Germany, this is a traditional recipe from the rural bourgeois cuisine and is therefore called Grandma's Lentil Stew.

This recipe makes about \emph{$6$-$8$ servings}. It takes about \emph{$45$ minutes} and has an \emph{easy} level of difficulty. 

\paragraph{Items}
\begin{itemize}[noitemsep]
	\item[\ding{182}] $\frac{1}{4}$ celery.
	\item[\ding{183}] $1$ leek.
	\item[\ding{184}] $1$ handful of parsley.
	\item[\ding{185}] $2$ carrots.
	\item[\ding{186}] $2$ slices of bacon.
	\item[\ding{187}] $500$g lentils.
	\item[\ding{188}] Salt and black pepper.
\end{itemize}

\paragraph{Instruction} 
