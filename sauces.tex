\chapter{Sauces}
\epigraph{``A sauce adds something, really two things: a taste as well as the opportunity to think about how the thing was made. This is the same kind of pleasure we derive when we look at a painting; the eye is pleased, while the mind explores the esthetic windings of a technique and a willed structure."}{--- \textup{Raymond Sokolov}}

\section{Mayonnaise}
\label{mayonnaise}
This is the classic recipe of a mayonnaise, refined in such a way that the basic taste of egg, mustard and oil is supported by citric acid and bitter substances from green tea. You will need a large bowl and a whisk, or optionally a hand blender.

This recipe makes \emph{$2$ servings}. It takes about \emph{$10$ minutes} and has an \emph{easy} level of difficulty.

\paragraph{Items}
\begin{itemize}[noitemsep]
    \item[\ding{182}] A pinch of green tea powder.
    \item[\ding{183}] $1$ teaspoon of dijon mustard.
    \item[\ding{184}] $1$ teaspoon of medium mustard.
    \item[\ding{185}] $1$ egg.
    \item[\ding{186}] $100$ ml sunflower oil.
    \item[\ding{187}] Lemon juice or a small lemon.
    \item[\ding{188}] Salt and pepper.
    
\end{itemize}

\paragraph{Instruction} To make the mayonnaise, first add the egg to our bowl. then add the juice of a small lemon half and the Dijon mustard as well as the medium hot mustard. We also add a pinch of green tea powder, salt and pepper.

Then we start whisking the ingredients and slowly add the sunflower oil. It is also possible to use rapeseed oil, but the oil should not have too pronounced taste of its own. If you use a hand-held dusting mixer for beating, it should be sunk over the egg yolk, like a protective dome over it, and then switched on at the highest speed. The operations are carried out until a solid paste is obtained, just the finished mayonnaise.

The mayonnaise can be steered in different directions with flavor nuances, such as a clove of garlic in the hearty spicy direction or a chili pepper towards the spicy one. Adding a bit of \emph{hoisin sauce} or \emph{sriracha sauce} also leads to an omami result or mild spiciness, respectively.

\section{Raspberry Chutney}
\label{raspberrychuttney}
The raspberry chutney is a base for sauces and can also be used as a burger sauce when combined with a mayonnaise [\ref{mayonnaise}]. However, it is also suitable as a stand-alone sauce for dark meat dishes, such as red or venison. Optionally, this recipe can also be done with \emph{blackberries}, \emph{blueberries}, \emph{gooseberries}, \emph{strawberries} or \emph{black cherries}. A winter version with \emph{oranges} or a summer version with \emph{bitter oranges} as a base is also possible.

This recipe makes \emph{$2$ servings}. It takes about \emph{$15$ minutes} and has an \emph{easy} level of difficulty. 

\paragraph{Items}
\begin{itemize}[noitemsep]
    \item[\ding{182}] $300$ g of raspberries.
    \item[\ding{183}] 1 onion.
    \item[\ding{184}] 1 tablespoon of brown sugar.
    \item[\ding{185}] 1 tablespoon of rice vinegar.
    \item[\ding{186}] 1 chilli.
    \item[\ding{187}] 2 tablespoons of olive oil.
    \item[\ding{188}] 3 tablespoons of soy sauce.
    \item[\ding{189}] Salt, pepper and cinnamon.
\end{itemize}

\paragraph{Instruction} We put on a medium size pot and add the olive oil. The oil should then be heated over medium heat. While the olive oil is heating up, we chop the chili pepper as small as possible and add it to the oil. A stewing time of about five to seven minutes should be allowed so that the oil can absorb the flavors of the chili. Here you can vary the degree of spiciness to your liking, but Thai chilies are best, as the chutney should be fiery at the end.

Then we add the chopped onions and let them become translucent. Once the state is reached, the brown sugar is added. We now increase the heat slightly and let the sugar caramelize. When the onions are covered with the caramel layer and as a result are golden brown, we add the raspberries. You can use fresh or frozen raspberries. Only the cooking time is delayed, because of the additional water in the case of the frozen option.

We can add the rice vinegar and soy sauce at the same time and let the chutney boil down over medium heat.

Lastly, we add a glass of tap water and reduce the heat. Stirring occasionally, we reduce the broth until we get a thick, creamy consistency. At last we taste with a little cinnamon, salt and pepper. We should save on the salt, because the soy sauce contains enough sodium.
