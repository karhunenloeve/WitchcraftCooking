\chapter{Sauces}
\epigraph{``A sauce adds something, really two things: a taste as well as the opportunity to think about how the thing was made. This is the same kind of pleasure we derive when we look at a painting; the eye is pleased, while the mind explores the esthetic windings of a technique and a willed structure."}{--- \textup{Raymond Sokolov}}

Sauces have played a pivotal role in European culinary history, enhancing the flavors of dishes and reflecting the diverse cultures and regions of the continent. This concise overview traces the evolution of sauces in Europe, highlighting key moments and influences.

\textbf{Ancient Greece and Rome:} The origins of European sauces can be traced back to ancient Greece and Rome. Both civilizations used sauces to enhance their foods. The Greeks, for example, created a variety of sauces using ingredients like olive oil, vinegar, and herbs. The Romans further developed this tradition, with cookbooks like "Apicius" detailing numerous sauce recipes, including garum, a fermented fish sauce.

\textbf{Medieval Europe:} During the Middle Ages, sauces underwent significant changes. The use of spices from the East, such as black pepper, cinnamon, and cloves, became increasingly popular in European cuisine. Sauces were used to mask the flavors of preserved foods and add complexity to dishes. One notable sauce from this period was "cameline sauce", made with cinnamon, ginger, and cloves, which graced many tables.

\textbf{Renaissance and the Age of Exploration:} The Renaissance marked a period of culinary refinement in Europe. Exploration and trade brought new ingredients, like tomatoes and chili peppers, to the continent. The arrival of tomatoes in the 16th century led to the creation of tomato-based sauces, which eventually became a cornerstone of Italian cuisine.

\textbf{17th and 18th Centuries:} The French played a pivotal role in shaping European sauce culture during the 17th and 18th centuries. Chef François Louis Franquet is credited with popularizing the "mother sauces" – bechamel, velouté, espagnole, and tomato – as foundational elements of French cuisine. These sauces served as the base for numerous derivative sauces, a concept still central to classical French cooking today.

\textbf{19th Century:} The 19th century witnessed the emergence of nouvelle cuisine in France, marked by a shift towards lighter, more refined sauces. Chef Auguste Escoffier played a key role in codifying many classic French sauces, such as hollandaise and béarnaise. Additionally, mustard and mayonnaise became staples in European kitchens, offering versatility in flavor enhancement.

\textbf{20th Century:} The 20th century saw a fusion of culinary influences from around the world. The British popularized Worcestershire sauce, a complex condiment of Indian origin. Russian dressing, a creamy sauce, was born in the United States. Ketchup, derived from Chinese fermented sauces, gained global popularity.

\textbf{Contemporary Era:} In the 21st century, European sauces continue to evolve, with an emphasis on innovation and sustainability. Traditional sauces are reimagined with modern twists, and health-conscious consumers seek lighter alternatives. Additionally, European chefs draw inspiration from global cuisines, creating fusion sauces that reflect multicultural influences.

\textbf{Regional Diversity:} Europe boasts a rich tapestry of regional sauces. Italy's pesto, France's bordelaise, Spain's romesco, and Germany's schnitzel sauce are just a few examples of how sauces reflect the unique flavors and ingredients of specific areas.

In conclusion, sauces have evolved significantly throughout European culinary history. From their humble origins in ancient Greece and Rome to the refinement of French classical sauces, sauces have always played a vital role in enhancing the flavors of European cuisine. Today, they continue to adapt and thrive, reflecting both tradition and innovation in the diverse kitchens of Europe.
\clearpage

\section{Mayonnaise}
\label{mayonnaise}
Within these pages, we present to you the timeless recipe for mayonnaise, meticulously refined to elevate the fundamental flavors of egg, mustard, and oil. Here, tradition marries innovation as we introduce the nuanced companionship of citric acid and the subtle bitterness of green tea. To embark on this culinary journey, equip yourself with a capacious bowl and a trusty whisk or, alternatively, consider employing a hand blender for added efficiency. In crafting this delectable mayonnaise, we delve into the realm of culinary alchemy, where the harmonious interplay of ingredients yields a symphony of taste. The intrinsic creaminess of egg, the zesty piquancy of mustard, and the unctuous richness of oil converge to form the backbone of this beloved condiment. As we venture further, we introduce citric acid, lending a bright, citrusy flourish to the mayonnaise's profile. This modest addition elevates the ensemble to new heights, awakening the palate with its invigorating acidity. Furthermore, we pay homage to the art of subtlety by incorporating bitter nuances derived from green tea. This infusion bestows an intriguing depth to the mayonnaise, a whisper of complexity that dances on the taste buds, creating a captivating contrast to the creamy base.

In terms of culinary implements, a generously sized bowl and a steadfast whisk are your traditional allies in this endeavor. Alternatively, for those inclined toward modern conveniences, a hand blender may be employed to expedite the process. As you embark on this culinary odyssey, rest assured that you are about to embark on a journey that marries tradition and innovation, resulting in a mayonnaise that is a veritable symphony of flavors, where every element plays a harmonious role in the culinary overture.

This recipe makes \emph{$2$ servings}. It takes about \emph{$10$ minutes} and has an \emph{easy} level of difficulty.

\paragraph{Items}
\begin{itemize}[noitemsep]
    \item[\ding{182}] A pinch of green tea powder.
    \item[\ding{183}] $1$ teaspoon of dijon mustard.
    \item[\ding{184}] $1$ teaspoon of medium mustard.
    \item[\ding{185}] $1$ egg.
    \item[\ding{186}] $100$ ml sunflower oil.
    \item[\ding{187}] Lemon juice or a small lemon.
    \item[\ding{188}] Salt and pepper.
    
\end{itemize}

\paragraph{Instruction}
In the pursuit of crafting the perfect mayonnaise, the journey begins by placing a single egg into our bowl. To this, we add the zestful juice of half a small lemon, alongside the robust notes of Dijon mustard and the medium heat of another mustard variant. As a subtle yet enchanting twist, a mere pinch of finely ground green tea powder, seasoned judiciously with salt and pepper, is gently integrated into the mix. With all elements thoughtfully assembled, we initiate the delicate choreography by commencing the whisking process, all the while introducing a slow drizzle of sunflower oil. Alternatively, the choice of rapeseed oil may be considered, provided it maintains a modest and unobtrusive flavor profile. If, in pursuit of expediency, a handheld immersion blender is enlisted, it should be lowered gently over the egg yolk, akin to a protective canopy, before being set to the highest speed. These meticulous actions persist until the amalgamation of ingredients transforms into a cohesive, velvety paste—an unequivocal testament to the completion of our exquisite mayonnaise.

This mayonnaise, while already a masterpiece in its own right, beckons culinary exploration and personalization. It is a versatile canvas upon which flavor nuances may be deftly painted. For those inclined toward the savory, a clove of garlic imparts a hearty, robust direction, while the inclusion of a chili pepper lends a fiery, piquant note. The addition of a trace of Hoisin sauce or Sriracha sauce, on the other hand, introduces an umami richness or a gentle spiciness, respectively, providing the discerning palate with a spectrum of tantalizing possibilities.
\clearpage

\section{Raspberry Chutney}
\label{raspberrychuttney}
The raspberry chutney serves as a versatile cornerstone for various sauces and can effortlessly transform into a delectable burger condiment when harmoniously blended with mayonnaise (as per reference \ref{mayonnaise}). Yet, its culinary prowess extends further, as it graciously lends itself to the role of a standalone accompaniment, ideally paired with rich, dark meat dishes such as those featuring succulent cuts of red meat or venison.

Intriguingly, this recipe holds the capacity for diverse adaptations. One may elect to replace raspberries with an array of alternatives, including blackberries, blueberries, gooseberries, strawberries, or black cherries, each contributing its unique character to the symphony of flavors. For those seeking a seasonal twist, a wintertime rendition incorporating oranges or a summery variation featuring the tangy essence of bitter oranges presents itself as an enticing option, expanding the culinary horizons and inviting the palate on a delightful journey of exploration.

This recipe makes \emph{$2$ servings}. It takes about \emph{$15$ minutes} and has an \emph{easy} level of difficulty. 

\paragraph{Items}
\begin{itemize}[noitemsep]
    \item[\ding{182}] $300$ g of raspberries.
    \item[\ding{183}] 1 onion.
    \item[\ding{184}] 1 tablespoon of brown sugar.
    \item[\ding{185}] 1 tablespoon of rice vinegar.
    \item[\ding{186}] 1 chilli.
    \item[\ding{187}] 2 tablespoons of olive oil.
    \item[\ding{188}] 3 tablespoons of soy sauce.
    \item[\ding{189}] Salt, pepper and cinnamon.
\end{itemize}

\paragraph{Instruction}
Commence by selecting a medium-sized pot and placing it over a medium flame. Pour the olive oil into the pot, allowing it to gradually warm within its confines. While the olive oil gently heats, shift your focus to the chili pepper. Aim for meticulous precision as you finely mince it, then incorporate it into the warming oil. A patient interval of five to seven minutes is imperative during this phase, affording the oil ample time to absorb the fiery essence of the chili. It's worth noting that the degree of spiciness can be customized to your personal preferences; nevertheless, Thai chilies are highly recommended for achieving the desired fiery crescendo in the chutney.

Following this, add the finely chopped onions to the pot, allowing them to sauté gently until they reach a state of translucent translucence. Once this transformation is accomplished, introduce the brown sugar to the pan. Gradually increase the heat, fostering the gradual caramelization of the sugar. Await the juncture when the onions are enveloped in a delicate caramel layer, their complexion evolving into a sumptuous shade of golden brown. Now, it is time to elegantly integrate the raspberries into this flavorful amalgamation. Whether opting for fresh or frozen raspberries, both are suitable choices. However, please bear in mind that the cooking duration may experience a slight variance in the case of frozen berries due to their added water content. Concurrently, introduce the rice vinegar and soy sauce to the ensemble. Permit the chutney to gently simmer over a medium flame.

As a final act of culinary alchemy, pour a glass of tap water into the pot, then reduce the heat. With intermittent stirring, the broth should steadily condense, yielding a luxuriously thick and creamy texture. In the closing moments, delicately season the chutney with a dash of cinnamon, judiciously applied salt, and a pinch of pepper. Exercise restraint with the salt, for the soy sauce already contributes a substantial sodium quotient to the concoction.