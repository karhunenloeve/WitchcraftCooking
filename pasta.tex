\chapter{Pasta}
\label{pasta}
\epigraph{``Everything you see I owe to Spaghetti."}{--- \textup{Sophia Loren}}

%\section{Cacio é Pepe}
%\section{Pasta alla Gricia}
\section{Spaghetti Carbonara}
\label{Spaghetticarbonara}
Few dishes spark as much controversy as Spaghetti Carbonara. Contrary to prevailing misconceptions, cream is not a constituent element of this dish; in fact, its inclusion is often met with disapproval. Carbonara can be executed with exquisite finesse using alternative pasta varieties, ideally Tagliatelle or Ziti. Nevertheless, we shall adhere to the traditional and most widely recognized version. The true challenge inherent in this culinary masterpiece lies in the intricacies of its preparation.

This recipe makes \emph{2 servings}. It takes about \emph{20 minutes} and has an \emph{intermediate} level of difficulty. 

\paragraph{Items}
\begin{itemize}[noitemsep]
    \item[\ding{182}] 250g of Spaghetti.
    \item[\ding{183}] 60g of Pecorino Romano or Pecorino Sardo.
    \item[\ding{184}] 60g Guanciale or Pancetta.
    \item[\ding{185}] 3 eggs.
    \item[\ding{186}] Salt and black pepper.
\end{itemize}

\paragraph{Instruction} 
As evidenced, this recipe calls for minimal ingredients. To commence, fill a generously sized pot approximately two-thirds full with water. A culinary dictum suggests that for every 100 grams of pasta, one liter of water is requisite. Proceed to season the water with three tablespoons of table salt, then set it to boil. While the water steadily reaches its boiling point, turn your attention to crafting the sauce. Begin by meticulously grating the Pecorino cheese into the finest granules possible using a square grater. At this juncture, you have the option to blend 30 grams of Pecorino with 30 grams of Parmesan, or adjust the ratio to your personal preference. It is worth noting, however, that Pecorino carries a notably saltier profile, imparting a spicier nuance to the dish.

To form the base for our sauce, place the finely grated cheese into a bowl. Now, introduce the three raw eggs into the mix. Alternatively, you may opt to use only the yolks at this stage, resulting in a distinct hue and a more robust flavor. By discarding the egg whites, you increase the relative fat content, providing a bolder platform for the Guanciale and Pecorino. Nevertheless, the original recipe adheres to the use of all ingredients. Proceed to vigorously whisk the cheese and eggs together until they amalgamate into a thick, moderately uniform mixture. At this point, generously add freshly ground black pepper to the concoction. The quantity of pepper remains at your discretion, and do not hesitate to be liberal with it. A bountiful peppery presence is imperative. With this, the foundation for our creamy sauce is now complete.

Now, it's time to introduce your chosen pasta—whether spaghetti, ziti, tagliatelle, or another delightful variety—into the simmering pot. Allow the pasta to cook for the duration specified on the package or, for an authentic "al dente" texture, reduce the cooking time by one minute. Pasta with a broad surface area is particularly well-suited for this dish.

While the pasta gracefully undergoes its transformation, proceed by slicing the Guanciale into slender, bite-sized strips. Simultaneously, heat a skillet over medium heat, gently introducing more freshly grated pepper to the pan—no additional fats required. Roast the pepper until its aromatic vapors tantalizingly titillate the senses, a sign that it's ready. Now, add the Guanciale to the pan and allow it to sizzle and crisp to perfection. During this process, a generous amount of flavorful fat will be released from the pork cheeks—this very fat serves as the vessel of taste for our dish and should be cherished. Feel free to taste for doneness, ensuring the Guanciale reaches that ideal crispy texture. Once achieved, turn off the heat entirely.

At this juncture, it's time to marry the pasta and the Guanciale. Retrieve a ladleful of pasta water and pour it into the pan, savoring the harmonious release of pepper and roasted flavors from the rendered fat. Your pasta should now be cooked to perfection. Gently lift it from the pot and deposit it into the pan, nestling it alongside the Guanciale and pepper. However, it's imperative to preserve the pasta water—it plays a crucial role in the final steps.

Pay close attention now, for this is where culinary artistry takes center stage. As the pasta intertwines with the Guanciale and pepper, it will emit steam, slowly losing its heat. The moment you discern that steam has become a mere whisper or has dissipated entirely, seize a quarter ladle of the reserved pasta water and incorporate it into the cheese mixture. This addition will yield a slightly more fluid cream. This luscious cream is then poured over the warm pasta, and with determination, you must vigorously amalgamate the components until they meld into a harmonious, creamy masterpiece.
\clearpage

\section{Al'Amatriciana}
\label{alamatriciana}
Pasta all'Amatriciana is a classic Italian dish that originates from the town of Amatrice in the Lazio region. This flavorful and satisfying pasta dish is known for its simple yet robust combination of ingredients. The star of the show is the rich tomato sauce, infused with the smoky and salty flavors of guanciale or pancetta. Tossed with al dente pasta and finished with a generous sprinkle of grated Parmesan or Pecorino cheese, Pasta all'Amatriciana is a true crowd-pleaser. Whether you're looking for a quick weeknight dinner or a dish to impress your guests, this traditional Italian recipe is sure to satisfy your cravings for a hearty and delicious meal.

This recipe makes \emph{2 servings}. It takes about \emph{15 minutes} and has an \emph{easy} level of difficulty. 

\paragraph{Items}
\begin{itemize}[noitemsep]
    \item[\ding{182}] 250g pasta.
    \item[\ding{183}] 150g Guanciale or optional Pancetta.
    \item[\ding{184}] 2 cans of date tomatoes.
    \item[\ding{185}] 50g of Pecorino Romano or optional Pecorino Sardo.
    \item[\ding{186}] Salt and pepper.
\end{itemize}

%\section{Pasta Norcina}
%\section{Pasta alla Puttanesca}
%\section{Pasta al Vino}
