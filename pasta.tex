\chapter{Pasta}
\epigraph{``Everything you see I owe to spaghetti."}{--- \textup{Sophia Loren}}

%\section{Cacio é Pepe}
%\section{Pasta alla Gricia}
\section{Spaghetti Carbonara}
\label{spaghetticarbonara}
There is hardly a more controversial dish than \emph{spaghetti carbonara}. Contrary to many opinions, cream is not a component of this dish. It is even frowned upon. The carbonara can be made very well with other pasta, ideally even with tagliatelle or ziti. However, we will stick to the most popular variant. The challenge with this dish is the process of preparation.

This recipe makes \emph{$2$ servings}. It takes about \emph{$20$ minutes} and has an \emph{intermediate} level of difficulty. 

\paragraph{Items}
\begin{itemize}[noitemsep]
    \item[\ding{182}] $250$g of spaghetti.
    \item[\ding{183}] $60$g of pecorino romano or pecorino sardo.
    \item[\ding{184}] $60$g guanciale or pancetta.
    \item[\ding{185}] $3$ eggs.
    \item[\ding{186}] Salt and black pepper.
\end{itemize}

\paragraph{Instruction} 
As we can see, we do not need many ingredients. First, we fill a large pot two-thirds with water. It is said that for every $100$g of pasta, one liter of water is needed. We salt the water with three tablespoons of table salt and bring it to a boil.

While the water is boiling, we prepare the sauce. To do this, we grate the pecorino as finely as possible on the square grater. At this point you can also mix $30$g pecorino with $30$g parmesan, or a ratio of your choice. Just keep in mind that the pecorino is much saltier and gives the dish a much spicier note. 

To create the cream, we put the grated cheese in a bowl. Now we add the three raw eggs. You can also use only the yolk at this point. The result is a different color of the sauce and a much more intense flavor. If you discard the egg whites, the relative fat content is much higher and you have a stronger flavor carrier for the guanciale and pecorino. However, the original recipe calls for the use of all the ingredients. The cheese is then whisked with the eggs until a thick, fairly homogeneous mass is obtained. Here is now added abundant black pepper, freshly ground. Subsequently, the mixture is whisked again. The amount of pepper I leave to you, but feel free to overdo it. It must be generous. Now the base for the cream is ready.

We now add the \emph{spaghetti}, \emph{ziti}, \emph{tagliatelle} or other form of pasta to the pot and cook it for the same amount of time as indicated on the package, or one minute shorter to get the pasta really all dente. Pasta with a large surface is particularly suitable.

We cut the guanciale into thin short strips. Next, we heat a skillet over medium heat and add to the pan, without the addition of any fats, again freshly grated pepper. We roast the pepper until the respiratory tract is irritated by the rising steam. Then we add the guanciale. We let it roast until the guanciale becomes crispy. A large amount of fat will come out of the pork cheeks, this fat is the flavor carrier for the dish and should not be discarded. Feel free to taste here to see if the guanciale is crispy enough. If it is, turn off the heat completely. Now we take a ladle of pasta water and add it to the pan to release the adhering pepper and roasted flavors from the fat. The pasta should be cooked, now. We skim it and put the finished cooked pasta in the pan with the guanciale and pepper. However, the pasta water should be saved! At this point it is necessary to watch carefully. The pasta will steam, but slowly cool from this point on. As soon as there is only a little steam, or even no steam at all, take a quarter of a ladle of pasta water and add it to the cheese to obtain a slightly thinner cream. We then add this cream to the warm pasta and mix vigorously.

The result is the classic carbonara cream, thus we are ready to serve.

%\section{Al'Amatriciana}
%\section{Pasta Norcina}
%\section{Pasta alla Puttanesca}
%\section{Pasta al Vino}
